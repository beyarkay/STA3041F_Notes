\title{Math Stats Proofs}
\author{ Boyd Kane }
\date{Last updated: \today}
\documentclass[12pt]{report}
% Allow images to be used
\usepackage{graphicx}
\graphicspath{{./images/}}
% Clickable hyperlinks
\usepackage{hyperref}
% Colourful equations
\usepackage[dvipsnames]{xcolor} 
% Nice math
\usepackage{amsmath} 
% Nice symbols in math
\usepackage{amssymb} 
% Required in order to calculate the width of a certain piece of text
\usepackage{calc}
% Make a \sim with iid on top of it
\newcommand\iid{\mathrel{\overset{\makebox[0pt]{\mbox{\normalfont\tiny\sffamily
iid}}}{\sim}}}
% Make the independence symbol with two perpendiculars next to each other
\newcommand{\indep}{\perp \!\!\! \perp}
% Explain math using an underbrace
\newcommand{\explain}[2]{\underbrace{#1}_{\parbox{\widthof{\ensuremath{#1}}}{\footnotesize\raggedright #2}}}

\begin{document}
\maketitle
\tableofcontents
\section{About this document}
This document does \textit{not} contain all the proofs required for STA3041F,
as we have not been given a list of proofs to learn. When I have found a proof
used in a past paper, I've listed the past paper in the margins with the year,
and a code like T2 for Test 2 or E for Exam.
\marginpar{Example note}
\subsection{PGF of a Branching Process}
Work in progress
\subsection{Expectation of a Branching Process}
Note that the following are true for any random variable $X$:
\begin{itemize}
    \item $ \mathbb{G}_{X}(1) = 1 $ 
    \item $ \mathbb{G}_{X}'(1) = \mathbb{E}\left[X\right]$ 
\end{itemize}
And so the expectation of a branching process goes like:
\begin{equation*}
    \begin{aligned}
        \mathbb{E}\left[Z_{m}\right] &= \left.\frac{d}{ds} \mathbb{G}_Z{_{m}}(s)
        \right|_{s=1}  \\
            &= \left.\frac{d}{ds}\mathbb{G}_X\left(\mathbb{G}_{Z_{m-1}}(s) \right)
            \right\vert_{s=1}\\
            &= \mathbb{G}_X'\left( \mathbb{G}_Z{_{m-1}}(1) \right)
            \cdot \mathbb{G}_Z{_{m-1}}'(1) \quad\text{(Chain Rule)}\\
            &= \mathbb{G}_X'\left( 1 \right) \mathbb{G}_Z{_{m-1}}'(1)\\
            &= \mathbb{E}\left[ X \right]  \mathbb{E}\left[Z_{m-1}\right] \\
            &= \mathbb{E}\left[ X \right]  \mathbb{E}\left[X\right] \mathbb{E}\left[Z_{m-2}\right] \\
            &\vdots \\
        \mathbb{E}\left[Z_{m}\right] &= \mathbb{E}\left[ X \right]^{m} \\
            &= \mu^{m} \\
    \end{aligned}
\end{equation*}
    

\subsection{Variance of a Branching Process}
Work in progress
\subsection{Probability of Eventual Extinction of a Branching Process}
There's two parts to this, 
\begin{enumerate}
    \item The part that $G(s) - s = 0$ is the probability of eventual
        extinction.
    \item The part that the probability of eventual extinction is the smallest
        positive root of $s$.
\end{enumerate}
\subsubsection{Proof of $G(s) - s = 0$}
Note that this isn't exactly how it's laid out in the notes. I've tried to
simplify the terminology a bit.
\begin{equation*}
    \begin{aligned}
        \eta &= \mathbb{P}\left[\bigcup_{m=1}^{\infty} Z_{m} = 0\right] \\
            &= \lim_{m\to\infty} \mathbb{P}\left[Z_{m} = 0\right] \\
            &= \lim_{m\to\infty} \mathbb{G}_{Z_{m}}(0) \\
            &= \lim_{m\to\infty} \mathbb{G}_{X}( \mathbb{G}_{Z_{m-1}}(0)) \\
            &= \lim_{m\to\infty} \mathbb{G}_{X}(\mathbb{P}\left[Z_{m-1} = 0\right]) \\
            &=  \mathbb{G}_{X}(\lim_{m\to\infty} \mathbb{P}\left[Z_{m-1} = 0\right])\\
            &=  \mathbb{G}_{X}\left(\mathbb{P}\left[\bigcup_{m=1}^{\infty}
            Z_{m} = 0\right]\right)\\
            \eta &=  \mathbb{G}_{X}(\eta)\\
    \end{aligned}
\end{equation*}

\subsubsection{Proof that the probability of eventual extinction is the
smallest positive root}
Given that the probability of eventual extinction $\eta$ is a root of $
\mathbb{G}_X(s) - s = 0$, we now prove that $\eta$ is the smallest non-negative
root, note that:
\begin{itemize}
    \item We define $w$ as a solution to $ \mathbb{G}_X(w) = w $
    \item The process, by definition starts out with positive population:
        $Z_{0} > 0 \implies \mathbb{P}\left[Z_{0} = 0\right] = 0$
    \item $\eta_{m} = \mathbb{G}_X(\eta_{m-1}) =
        \mathbb{G}_X(\mathbb{G}_X(\eta_{m-2})) = \dots $
    \item The generating function has infinite positive terms, so
        is non-decreasing: $\mathbb{G}_X(s) \le \mathbb{G}_X(t) \implies s \le t$
\end{itemize}
So, to actually do the proof:
\begin{equation*}
    \begin{aligned}
        \eta_{0} &= \mathbb{P}\left[Z_{0} = 0\right] = 0 \le w \\
        \eta_{0} &\le w \\
        \mathbb{G}(\eta_{0}) &\le \mathbb{G}(w) = w \quad\text{(by definition
        of $w$)} \\
        \eta_{1} &\le w \\
        \mathbb{G}(\eta_{1}) &\le \mathbb{G}(w) \\
        \eta_{2} &\le w \\
        \mathbb{G}(\eta_{2}) &\le \mathbb{G}(w) \\
        \eta_{3} &\le w \\
                             &\vdots\\
        \text{So } \eta_{n} &\le w \quad \text{for all } n\\
        \eta &= \lim_{m \to \infty} \eta_{m} \le w
    \end{aligned}
\end{equation*}
So $\eta$ will always be smaller than any other possible root $w$ to the
equation $ \mathbb{G}(s) = s$

\subsection{MGFs for First Passage and First Return}
Work in progress
\subsection{Chapman-Kolmogorov Equations}
\marginpar{2019 T1}
Given that a chain has the Markov property, the equations are, for all $0 \le v
\le t$:
\begin{equation*}
    \begin{aligned}
        p_{ij}(s, t) :&= \mathbb{P}\left[X_{s+t} = j | X_{s} = i\right] \\
                     &= \sum_{k\in\mathcal{U}} p_{ik}(s,v) \cdot p_{kj}(s+v,t-v)\\
    \end{aligned}
\end{equation*}
The proof is due to the law of total probability and the Markov property. It
might help to first recall that:
\begin{equation*}
    \begin{aligned}
        \mathbb{P}\left[A | B\right] &= \frac{ \mathbb{P}\left[A \cap B\right]}{ \mathbb{P}\left[B\right] } \\
            &\iff \\
        \mathbb{P}\left[A \cap B \right]  &=  \mathbb{P}\left[A | B\right] \cdot \mathbb{P}\left[B\right] \\
    \end{aligned}
\end{equation*}
So the proof is:
\begin{equation*}
    \begin{aligned}
        p_{ij}(s, t) &= \mathbb{P}\left[X_{s+t} = j | X_{s} = i\right] \\
            &= \sum_{k\in\mathcal{U}} \mathbb{P}\left[X_{s+t} = j, X_{s+v} = k | X_{s} = i\right]  \\
            &= \sum_{k\in\mathcal{U}} \mathbb{P}\left[ X_{s+v} = k | X_{s} = i \right] \cdot \mathbb{P}\left[ X_{s+t} = j | X_{s+v} = k, X_{s} = i\right]  \\
            &= \sum_{k\in\mathcal{U}} \mathbb{P}\left[ X_{s+v} = k | X_{s} = i \right] \cdot \mathbb{P}\left[ X_{s+t} = j | X_{s+v} = k\right]  \\
            &= \sum_{k\in\mathcal{U}} p_{ik}(s,v) \cdot p_{kj}(s+v,t-v)\\
    \end{aligned}
\end{equation*}

\subsection{Probability of one Exponential RV being less than another}
\marginpar{2019 T1}
Where $X_{i} \iid Exp(\lambda_{i})$ for $i \in \{1, 2\}$:
\begin{equation*}
    \begin{aligned}
        \mathbb{P}\left[X_{1} < X_{2}\right] &= \int_{0}^{\infty} \int_{0}^{x_{2}} \left( \lambda_{1}e^{-\lambda_{1}x_{1}} \right) \cdot \left( \lambda_{2}e^{-\lambda_{2}x_{2}} \right) \,dx_{1}  \,dx_{2}  \\
            &= \int_{0}^{\infty} \int_{0}^{x_{2}}  \lambda_{1}e^{-\lambda_{1}x_{1}} \,dx_{1} \, \lambda_{2}e^{-\lambda_{2}x_{2}}  \,dx_{2}  \\
            &= \int_{0}^{\infty} \left( 1 - e^{-\lambda_{1}x_{2}}\right) \cdot \left( \lambda_{2}e^{-\lambda_{2}x_{2}} \right) \,dx_{2}  \\
            &= 1 - \lambda_{2} \int_{0}^{\infty} e^{-(\lambda_{1} + \lambda_{2})} \,dx_{2} \\
            &= \frac{\lambda_{1}}{\lambda_{1} + \lambda_{2}} \\
    \end{aligned}
\end{equation*}

\end{document}

